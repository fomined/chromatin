\documentclass{article}
\usepackage[utf8]{inputenc}
\usepackage[russian]{babel}
\begin{document}
\title{Алгоритмизация в области исследований структуры хроматина}
\maketitle
\section{Введение}
текст введения
\section{Постановка задач}
\subsection{Задача №1}
Дана закольцованная цепь (начало соединено с концом)
\\$N$ - количество элементов в цепи
\\Требуется найти минимальную укладку цепи на двумерную сетку
\section{Решение задач}
\subsection{Задача №1}
\subsubsection{Перебор с возвратом}
Введем переменные:
\\$Exits = \{up,left,down,right\} =\{0,1,2,3\}$  - список из направлений
\\$E[i]\epsilon Exits$, $i\epsilon[0,N-1]$ - матрица размера $N$, содержит информацию о том, где на сетке расположен следующий элемент цепи относительно текущего. 
\\$A[i,j]\epsilon[-1,N+1]$ - матрица размера $(N+2)*(N+2)$, $0$ обозначет что клетка свободна, $-1$ - эта клетка является границей матрицы, другие числа - порядковый номер элемента цепи, уложенного в клетку сети по данным координатам
\\$k\epsilon[0,N-1]$ - номер текущего рассматриваемый элемент 
\\$x,y$ - координаты текущего элемента
\\
\\Заполним границу матрицы числами "$-1$"
\\Далее алгоритм:
\begin{verbatim}

//place first ball
k = 1
E[k] = 0
x = N/2
y = N/2
A[x,y] = 1
//try to place all another balls
while (true){
 //step -->
 switch(E[k]){
   0: y = y+1 //UP
   1: x = x-1 //LEFT
   2: y = y-1 //DOWN
   3: x = x+1 //RIGHT
 }
 //check next grid cell
 if (A[x,y]==0){
  if (k<=N){
    stepDo = true
  }
  else {
    stepDo = false
  } 
 }
 else {
  if (A[x,y]<0) OR (A[x,j]>1){
    stepDo = false
  }
  else { 
   //so, we at start, A[x,y]=1
   if(k==N){
     //SOLUTION!
     //Do some staff about it     
   }     
   // here no way to go ahead in any case
   stepDo = false     
  }
 }
 //choose step forward / change direction / step backward
 if(stepDo){ 
  k = k + 1
  A[x,y] = k
  E[k] = 0
 }
 else {
   //step <-- 
   stepBack = true
   while(stepBack){ 
     switch(E[k]){
       0: y = y-1 //UP reverse
       1: x = x+1 //LEFT reverse
       2: y = y+1 //DOWN reverse
       3: x = x-1 //RIGHT reverse
     }
     E[k] = E[k] + 1
     if (E[k]<4){       
       stepDo = true       
       stepBack = false
     }
     else {
       A[x,y] = 0
       k = k - 1  
       if (k == 1) {
        // Last step ever
        // Nothing else to do
        // EJECT
        exit
       }                      
     }
   }  
 }
  
}
   
\end{verbatim}
\section{Заключение}
текст 
\end{document}