%\documentclass[tikz,a4paper]{article} % стандартный документ
\documentclass[tikz,a4paper]{scrartcl} % документ с подзаголовком
\setkomafont{disposition}{\normalfont\bfseries} % шрифт (под)заголовков

\usepackage[russian]{babel} % русский текст
\usepackage[utf8]{inputenc} % русский в кодировке UTF8
\usepackage{amssymb,amsfonts,amsthm,amsmath} % математика
\usepackage{graphicx,graphics} % рисунки
\usepackage{cite} % расширенный тип цитирования
\usepackage[square,sort,comma,numbers]{natbib} % расширенный тип библиографии
\usepackage{authblk} % форматирование имeн автора и аффилиации
\usepackage{pgfplots} % графики функций

\usepackage{tikz} % диаграммы
\usetikzlibrary{positioning,shapes,arrows,matrix}
\usetikzlibrary{decorations.pathmorphing,backgrounds,fit,petri}
\usetikzlibrary{calc,intersections,through,backgrounds}

\usepackage{geometry} % установка полей документа
\geometry{left=3cm,right=3cm,top=3cm,bottom=3cm}

\begin{document}

\title{Алгоритм эффективной генерации цепочек на сетке}
\subtitle{/техническое задание/}
\author{Э.~С.~Фомин}
\date{} % не печатать дату генерации документа
\maketitle
\large % размер фонта всего документа


\section*{Постановка задачи}
\textit{Задача}. Дана цепь длины $L$, где $L$ находится в диапазоне $[128, 1024]$ элементов. Цепь разбита на $N$ последовательных участков длины $L_i$, $\sum_i^N L_i = L$. Каждый участок последовательно пронумерован. Цепь складывается в пространстве в некоторую компактную\footnote{Смысл понятия <<компактность цепи>> определяется далее.} структуру. Полученная структура исследуется на контакты ее участков с получением таблицы контактов. Таблица контактов является двумерной, и содержит число контактов для любой пары участков. Необходимо, имея только таблицу контактов, восстановить структуру цепи. Предполагается, что задача имеет большое множество решений, потому целью задания является эффективная генерация множества решений, удовлетворяющих заданным ограниченим по контактам.

Требуется реализовать несколько вариантов задачи, отличающихся пространством поиска, типом сетки и таблицы контактов.
\begin{itemize}
\item пространство поиска является двумерным (сетка на плоскости) и трехмерным (сетка в объеме),
\item сетка должна быть задана как BFM сетка (bond fluctuation model) \cite{BMF1994}, поскольку построение цепи именно по такому типу сетки обеспечивает большую гладкость,\footnote {Экспериментальные версии кода могут использовать равномерные сетки.}
\item таблица контактов должна рассматриваться как таблица с целыми, так и с булевыми значениям, $U_{ij} = 0 \Rightarrow false$ и $U_{ij} > 0 \Rightarrow true$.
\end{itemize}

\section*{Предполагаемый вариант решения}
Эффективность простого перебора с возвратами неприемлима для решения задачи с заданной длиной цепочки, поскольку его вычислительная сложность оценивается как $3^L$ для равномерной сетки на плоскости, или $\geq 5^L$ для сетки в пространстве. Потому \textcolor{red} {метод перебора должен быть исключен}. Предлагаемый вариант решения основан на генерации цепочек вариантом метода Монте-Карло, в котором вероятность того или иного направления роста цепи зависит как и от локального окружения точки (число соседей), так и расстояния до точки контакта. Предполагается, что корректный подбор функции оценки для выбора направления роста цепи позволит с большой вероятностью выполнять генерацию цепей, удовлетворяющих как требованиям компактности, так и ограниченим по контактам.

\subsection*{Как выбирать направление роста цепи}
Направление роста цепи зависит от многих факторов, причем наиболее существенными из них являются:
\begin{enumerate}
\item модель цепи, которая отражает основные особенности моделируемого объекта (жесткость к изгибу, жесткость к кручению, склонность к изломам, однородность жесткости вдоль цепи), 
\item модель укладки цепи (компактность, разреженность, фрагментарность,  кластеризация, наличие мотивов укладки и прочее), 
\item текущие стерические ограничения (исключить для роста занятые ячейки пространство), 
\item необходимость иметь контакты с участками уже построенной цепи.
\end{enumerate}
И все эти факторы должны быть корректно, то есть, как минимум в первом приближении точности, заложены в функцию, которая выбирает направление дальнейшего роста цепи (<<функция выбора>>). Заметим, что методологически проще выбирать разные функции выбора для разных моделей цепи и разных типов укладки. И наоборот, прочие требования - отсутствия стерических затруднений и обеспечения контактов - должны учитываться в любой функции выбора. 

\begin{quote} \textit{Отступление для понимания}. 

Процесс выбора неформально может быть описан следующим образом. Допустим, мы имеем цепь, жесткую к изгибам и мягкую к кручению, однородную и не имеющую склонности к изломам (это наиболее естественный вариант цепи для предметной области). Это означает, что каждый следующий сегмент цепи $i$ по своему направлению достаточно мало отличается от направления предыдущего сегмента $i-1$ (жесткость к изгибам). При этом двугранный угол между плоскостями, образуемыми сегментами ($i-1$, $i$) и ($i$, $i+1$) меняется хаотическим образом вдоль цепи (свободно крутящаяся цепь). Вопрос. Как для такой цепи работает функция выбора?

Функция выбора должна учесть, что точки сетки, которые находятся в направлении последнего построенного сегмента цепи $\bold r_{i,i+1}$, должны иметь приоритет перед точками в стороне или ведущими цепь в обратном направлении. Причем выбор любой точки из тех, что находятся на окружности вокруг вектора $\bold r_{i,i+1}$, должен быть равновероятен (цепь является свободно крутящейся). 

Расставив вероятности выбора точек, зависящие от модели цепи, функция выбора должна учесть модели укладки, и соответствено скорректировать вероятности точек. Например, для модели плотной упаковки вероятности точек, которые локализованы рядом с уже построенной цепью должны быть выше, а для точек, рядом с которыми нет ни одной занятой точки, вероятность выбора должна быть близка к нулю.

Учет стерических затруднений выполняется наиболее просто, вероятности выбора точек, которые уже заняты цепью, обнуляются.

И последнее. Допустим, что в цепи есть контакт $AB$, и элемент цепи $A$ уже построен, а элемент $B$ еще нет. Функция выбора в этом случае должна повышать приоритет у точек, которые находятся в сторону элемента $A$, чтобы в конце концов свести точку цепи $B$ к $A$.

\end{quote}

Разберем факторы, влияющие на рост цепи подробно.
\subsubsection*{Модели цепи}
Требуется реализовать две модели цепи - однородную и неоднородную. Обе модели являются параметрическими, и учитывают жесткость по сгибу и кручению параметрами $k_b$ и $k_t$, соответственно. 

\textit{Однородная модель} характеризуется тем, что параметры жесткости $k_b$ и $k_t$ вдоль цепи не меняются. Также в однородной модели нет склоности к изломам цепи, и этот фактор игнорируется.

\textit{Неоднородная модель} характеризуется тем, что элементы цепи имеют новый атрибут, называемый состоянием. Число возможных состояний элементов $N_s$ фиксировано, $N_s \lesssim 10$. Все элементы цепи находятся в одном из $N_s$ состояний. Состояния элементов не вычисляются, они заданы во входных данных. Для каждого состояния определены свои значения жесткости $k_b$ и $k_t$. Благодаря тому, что в цепи есть участки с низкой жесткостью, в модели возможны резкие изломы цепи.

\subsubsection*{Модели упаковки}
Требуется реализовать модель плотной (компактной) упаковки. Плотная упаковка означает, что цепь на сетке размещается таким образом, что свободных ячеек внутри области, охваченной цепью, достаточно мало. Плотность упаковки может меняться в разных точках сетки $\bold r$ и характеризуется параметром $\rho(\bold r)$, который вычисляется следующим образом. Некоторая точка сетки откружается кубом с длиной ребра $l$, и рассчитывается $n_a(l)$ - полное число ячеек в кубе, $n_a(l) \sim l^3$, и $n_f(l)$ - число занятых ячеек в кубе. Локальная плотность упаковки в точке $\bold r$ определяется как 
\[\rho(\bold r) = \lim_{l\rightarrow 0} \frac{n_f(l)}{n_a(l)} \]
Полная плотность упаковки вычисляется как среднее значение плотности по всем точкам, через которые проходит цепь, 
\[\rho = \frac{\sum \rho(\bold r)\delta_c(\bold r)}{\sum \delta_c(\bold r)} = \frac{1}{L} \sum_{\bold r} \rho(\bold r)\delta_c(\bold r) \]
В последней формуле выполняется суммирование по всем точкам $\bold r$ сетки.  Функция $\delta_c(\bold r)$ характеризует захват цепью $c$ точки сетки $\bold r$, и она равна 1 или 0 в зависимости от того, захвачена или нет точка. 

 
\begin{thebibliography}{99}
\bibitem{BMF1994} http://en.wikipedia.org/wiki/Bond\_fluctuation\_model
\end{thebibliography}

\end{document}
